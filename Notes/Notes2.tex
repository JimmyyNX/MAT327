\section{Day 2: Bases and Sub-bases}
\begin{proposition}
    Let $\mathscr{B}$ and $\mathscr{B}'$ be bases for topologies $\tau$ and $\tau'$ respectively on $X$. Then the following are equivalent.
    \begin{itemize}
        \item $\tau'$ is finer than $\tau$
        \item for all $x\in X$ and for all $B\in\mathscr{B}$. if $x\in B$, then there exists $B'\in\mathscr{B}'$ such that $x\in B'\subseteq B$
    \end{itemize}
\end{proposition}
\begin{proof}
    Let $x\in X$ and $B\in\mathscr{B}\subseteq\tau$ so $B$ is open with respect to $\tau$, since $\tau'$ is finer than $\tau$, we have that $B\in\tau'$, and therefor there is some $B'\in B'$ such that $x\in B'\subseteq B$. 

Conversely let $U\in\tau$ be arbitrary. Since $U\in\tau$, we have that 
\[U=\bigcup_{\alpha\in\Lambda}B_\alpha\]
for some collection $B_\alpha\in\mathscr{B}$. We further have that each $B_\alpha=\bigcup_{\gamma\in\Lambda_\alpha}B'_\gamma$ with $B'_\gamma\in\mathscr{B}$. Then rewriting $U$ yields
\[U=\bigcup_{\alpha\in\Lambda}\left(\bigcup_{\gamma\in\Lambda_\alpha}B'_\gamma\right)\]
Thus $U$ is a union of elements from $\mathscr{B}'$, so $U\in\mathscr{B}'$
\end{proof}
Recall from last lecture the topology generated by the bases of open disks and open rectangles of $\mathbb{R}^2$, using this proposition we may show that both topologies are equivalent, as we may always find sufficiently small circles and rectangles that fit within other circles and rectangles.

\begin{example}[The Line]
    The underlying set is $\mathbb{R}$ and $\tau$ is generated by the open intervals $(x,y)$
\end{example}
\begin{example}[Sorgenfrey Line]
    This is also commonly refered to as the \href{https://en.wikipedia.org/wiki/Lower_limit_topology}{Lower Limit Topology}, the underlying set is $\mathbb{R}$ and $\tau$ is generated by clopen intervals $[x,y)$
\end{example}
\begin{example}[K-Topology]
    The \href{https://en.wikipedia.org/wiki/K-topology}{K-Topology} has an underlying set of $\mathbb{R}$ and $\tau$ is generated by
    \begin{align*}
        &(x,y) &(x,y)\backslash K
    \end{align*}
    where $K=\{\frac{1}{n}\vert n\in\mathbb{N}\}$
\end{example}
An exercise left by the prof is to show that the basis generating the K-topology is indeed a basis.
\subsection{Sub-Bases}
\begin{definition}[Sub-Bases]
    A \href{https://en.wikipedia.org/wiki/Subbase}{Sub-Bases} on $X$ is a collection, $\mathscr{S}$ of subsets of $X$ such that $\mathscr{S}$ covers $X$
\end{definition}
\begin{lemma}
    The finite intersection of elements of a sub-bases, $\mathscr{S}$, for a topology on $X$ is a bases for a topology calle dthe basis generated by $\mathscr{S}$
\end{lemma}
\begin{proof}
    Let $\mathscr{B}$ bet he collection of finite intersection of elements of $\mathscr{S}$, to prove $\mathscr{B}$ is a basis we must check that it covers $X$, which is true because $\mathscr{S}\subseteq\mathscr{B}$ and $\mathscr{S}$ covers $X$. and we may find a basis element inside the intersection of two elements. to do this we let $B_1,B_2\in\mathscr{B}$ be arbitrary, let $x\in B_1\cap B_2$, by definition, we have that $B_1$ and $B_2$ are a finite intersection of elements from $\mathscr{S}$, so $B_1\cap B_2$ is also a finite inetrsection of elements of $\mathscr{S}$, so picking $B_3=B_1\cap B_2\in\mathscr{B}$, it follows that
\[x\in B_3= B_1\cap B_2\subseteq B_1\cap B_2\]
Thus verifying that it is a bases
\end{proof}
