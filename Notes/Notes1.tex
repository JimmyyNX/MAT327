\section{Day 1: Introduction, Topological Spaces and Bases}
This class is taught by Daniel Wilches Calderon during the Summer 2024 semester. We will mostly be following Topology by Munkres, with some suggested readings on the Quercus page
\subsection{Marking Scheme}
\begin{enumerate}
    \item Final Exam - 30\%
    \item Term Test - 20\%
    \item Quizzes - 50\% (Best 8 of 10)

    Every week we will be given a problem set, and during tutorial one of those questions will be tested
\end{enumerate}
\subsection{Open sets in the Real Numbers}
Recall that $U\in\mathbb{R}$ is open if
\begin{center}
    for all $x\in U$, there exists $\varepsilon>0$ such that $x\in (x-\varepsilon,x+\varepsilon)\subseteq U$
\end{center}
Open sets have some properties. That the arbitrary union and finite intersection of open sets is still open
\begin{enumerate}
    \item Arbitrary Union

    Let $\{U_\alpha \vert \alpha\in\Lambda\footnote{We use $\Lambda$ to denote an arbitrary indexing set}\}$ be an arbitrary collection of open sets, if $x\in \bigcup_{\alpha\in\Lambda}$, then $x\in U_\alpha$ for some $\alpha\in\Lambda$, and since $U_\alpha$ is open, we have that there exists some $\varepsilon>0$ such that
    \[x\in(x-\varepsilon,x+\varepsilon)\subseteq U_\alpha \subseteq\bigcup_{\alpha\in\Lambda}U_\alpha\]
    This same result doesn't hold true for arbitrary intersection, but it does for finite.
    \item Finite intersection
    
    Let $\{U_k\vert k\leq n\}$ be a finite collection of open sets. if $x\in\bigcap_{k\leq n}U_k$, then $x\in U_k$ for all $k\leq n$. Since each $U_k$ is open, there are $\varepsilon_i>0$ such that $x\in(x-\varepsilon_i,x+\varepsilon_i)\subset U_i$ for $i=1,2,\ldots,n$. Then pick $\varepsilon = \min\{\varepsilon_1,\ldots,\varepsilon_n\}$, then it follows that
    \[x\in(x-\varepsilon,x+\varepsilon)\subseteq(x-\varepsilon_k,x+\varepsilon_k)\subseteq U_k\]
    for all $k\leq n$, so $(x-\varepsilon,x+\varepsilon)\subseteq\bigcap_{k\leq n}U_k$.
\end{enumerate}
\subsection{Topological Spaces}

\begin{definition}[Topological Space]
    A toplogy, $\tau$, on $X$ is a collection of subsets of $X$ that satisfy
    \begin{enumerate}
        \item $X,\varnothing\in\tau$
        \item $\tau$ is closed under arbitrary Union
        \item $\tau$ is closed under finite intersection
    \end{enumerate}
    The double $(X,\tau)$ is called a \href{https://en.wikipedia.org/wiki/Topological_space}{Topological Space}. Furthermore, the elements of $\tau$ are called open sets, and the elements of $X$ are called points.
\end{definition}
We first notice that showing the intersection of 2 sets of $\tau$ is still in $\tau$ is equivalent to showing the finite intersection of sets of $\tau$ will still be in $\tau$. This is true by an inductive argument.

The following are some examples of topologies on a set with 2 elements, the prof briefly went over the topologies on a set with 3 elements, but basically left it as an exercise.
\begin{example}[Topologies on 2 Point Sets]
    Let $X=\{x,y\}$, we have the following are possible topologies on $X$
    \begin{enumerate}
        \item $\{X,\varnothing\}$
        
        This is often called the indiscrete/trivial topology
        \item $\{X,\varnothing,\{x\},\{y\}\}$
        
        This is often called the discrete topology

        \item\href{https://en.wikipedia.org/wiki/Sierpi%C5%84ski_space}{Sierpinski Spaces}
        
        The main characteristic of these spaces are that one point is open, while the other is not. More explicitly, the following are the Sierpinski Spaces
        \begin{enumerate}
            \item $\{X,\{x\},\varnothing\}$
            \item $\{X,\{y\},\varnothing\}$
        \end{enumerate}
    \end{enumerate}
\end{example}
    You might've noticed that we labelled some of those examples as the \href{https://en.wikipedia.org/wiki/Discrete_space#:~:text=The%20discrete%20topology%20is%20the,set%20in%20the%20discrete%20topology.}{Discrete Topology} and the \href{https://en.wikipedia.org/wiki/Trivial_topology}{Indiscrete/Trivial} Topology.
\begin{definition}[Discrete and Indiscrete Topology]
    Given a set $X$, the discrete topology on $X$ is equal to the Power Set of $X$, and the indiscrete topology is equal to $\{x,\varnothing\}$
\end{definition}
    
\begin{definition}[Cofinite Topology]
    Let $X$ be a set, the \href{https://en.wikipedia.org/wiki/Cofiniteness#Cofinite_topology}{Cofinite Topology} on $X$ is the collection
    \[\tau=\{U\subseteq X\vert X\backslash U\text{ is finite}\}\cup\{\varnothing\}\]    
\end{definition}
Now to prove that this is a topology, we must prove 3 things. The entire set and the empty set exist within $\tau$. $\tau$ is closed under arbitrary union and finite intersection. The first of these is almost immediate as $\varnothing\in\tau$ by definition, and $X\backslash X=\varnothing$, which is a finite set.
\begin{enumerate}
    \item Arbitrary Union
    
    Let $\{U_\alpha\vert\alpha\in\Lambda\}$ be open subsets of $X$, we have that $X\backslash U_\alpha$ is open. Then notice that \[X\backslash\left(\bigcup_{\alpha\in\Lambda}U_\alpha\right) = \bigcap_{\alpha\in\Lambda}\left(X\backslash U_\alpha\right)\]
    by DeMorgan's, which is an intersection of finite sets, so it must also be finite.
    \item Finite Intersection
    
    Let $\{U_k\vert k\leq n\}$ be open, we have that $X\backslash U_k$ are finite, so then we notice that
    
    \[X\backslash\left(\bigcap_{k\leq n}U_k\right) = \bigcup_{k\leq n}\left(X\backslash U_k\right)\]
    By DeMorgan's, which is a finite union of finite sets, which implies that it is finite, implying it must be in $\tau$
\end{enumerate}
\begin{definition}[Cocountable Topology]
    The \href{https://en.wikipedia.org/wiki/Cocountable_topology}{Cocountable Topology} is defined synonymously to the cofinite topology. It is the collection 
    \begin{center}
        $\tau=\{U\subseteq X\vert X\backslash U$ is countable $\}\cup\{\varnothing\}$
    \end{center}
\end{definition}
Prove that this is a topology
\begin{definition}
    Let $\tau,\tau'$ be topologies on $X$, if $\tau\subseteq\tau'$, then we say 
    \begin{itemize}
        \item $\tau$ is coarser than $\tau'$
        \item $\tau'$ is finer than $\tau$
    \end{itemize}
    As a more general statement, if $\tau\subseteq\tau'$ or $\tau'\subseteq\tau$, then we say that $\tau$ and $\tau'$ are compareable.
\end{definition}
\subsection{Bases}
These are very similiar to the bases you're probably familiar with from linear algebra, where the linear combinations form the vector space, where as these bases are sets that ``Unionize'' to form the topological space.
\begin{definition}[Bases]
    A bases for a topology on a set $X$ is a collection fo subsets of $X$, $\mathscr{B}$ such that
    \begin{itemize}
        \item for all $x\in X$, there exists $B\in\mathscr{B}$ such that $x\in B$
        \item if $x\in B_1\cap B_2$, then there exists $B_3\in\mathscr{B}$ such that $x\in B_3\subseteq B_1\cap B_2$
    \end{itemize}
\end{definition}
\begin{lemma}[Generated Topologies]
    Let $\mathscr{B}$ be a basis of $X$, then the collection of all subsets $U$ of $X$ such that for all $x\in U$ there exists $B\in\mathscr{B}$ such that $x\in B\subseteq U$ is a topology on $X$. This is called the \href{https://proofwiki.org/wiki/Definition:Generated_Topology#:~:text=Definition%202-,The%20topology%20generated%20by%20S%2C%20denoted%20%CF%84(S)%2C,element%20of%20S%20is%20open.}{Generated Topology}    
\end{lemma}
\begin{proof}
    We now need to prove that this is indeed a topology. We first have that $\varnothing\in\tau$ for vacuous reasons, then we also have that $X\in\tau$ because by letting $x\in X$ be arbitrary, we have that, by definition, there exists a $B\in\mathscr{B}$ such that $x\in B\subseteq X$.
\begin{enumerate}
    \item Union
    
    Let $\{U_\alpha\vert\alpha\in\Lambda\}$ be some collection of open sets, let $x\in\bigcup_{\alpha\in\Lambda}U_\alpha$, then there is some $\alpha\in\Lambda$ such that $x\in U_\alpha$, then let $B\in\mathscr{B}$ be such that $x\in B\subseteq U_\alpha\subseteq \bigcup_{\alpha\in\Lambda}U_\alpha$

    \item Intersection
    
    Let $U_1,U_2\in\tau$, let $x\in U_1\cap U_2$, since $U_1,U_2\in\tau$, there exists $B_1,B_2$ such that
    \begin{itemize}
        \item $x\in B_1\subseteq U_1$
        \item $x\in B_2\subseteq U_2$
    \end{itemize}
    so there must exist $B_3$ such that $x\in B_3\subseteq B_1\cap B_2$
\end{enumerate}
\end{proof}
\begin{example}[Circles]
    The bases of open disks in $\mathbb{R}^2$
    \[\tau = \{\{x\in\mathbb{R}^2\vert d(x,y)<\varepsilon\}\vert y\in\mathbb{R}^2,\varepsilon>0\}\]
\end{example}
\begin{example}[Rectangles]
    \[\tau=\{(a_1,b_1)\times(a_2,b_2)\vert a_1,b_1,a_2,b_2\in\mathbb{R}\}\]
\end{example}
\begin{lemma}
    For subset $\mathscr{B}$ of $\tau$ on $X$, The following are equivalent
    \begin{itemize}
        \item $\mathscr{B}$ is a basis generating $\tau$
        \item every non-empty element of $\tau$ is a union of elements of $\mathscr{B}$
    \end{itemize}
\end{lemma}
\begin{proof}
    Letting $(X,\tau)$ be a topological space, we begin with the forwards direction. Let $U\in\tau$ and let $x\in U$, since $\tau$ is generated by $\mathscr{B}$, there exists some $B_x\in\mathscr{B}$ such that $x\in B_x\subseteq U$, so it follows that
    \[U=\bigcup_{x\in U}B_x\footnote{This equality wasn't immediately obvious to me why this is true, but you can prove it with double inclusion}\]
    Conversely, let $x\in X$ be arbitrary. Since $X\in\tau$, $X=\bigcup_{\alpha\in\Lambda}U_\alpha$ with $B_\alpha\in\mathscr{B}$ it follows that $x\in B_\alpha$ for some $\alpha\in\Lambda$. Now let $B_1,B_2\in\mathscr{B}$ with $x$ existing in their intersection. We first have that $B_1\cap B_2$ is open, so we may represent it as a union,
    \[ B_1\cap B_2 = \bigcup_{\alpha\in\Lambda}B_\alpha\] so $x\in B_\alpha$ for some $\alpha\in\Lambda$. so $x\in B_\alpha\subseteq B_1\cap B_2$ 
\end{proof}
