\section{Day 3 : Axiom of Choice, Zorn's Lemma and Ordering}
\subsection{Axiom of Choice}
\begin{theorem}[\href{https://en.wikipedia.org/wiki/Axiom_of_choice}{Axiom of Choice}]
    if $\{A_\alpha\vert \alpha\in\Lambda\}$ is a collection of non-empty sets, then their product
    \[\prod_{\alpha\in\Lambda}A_\alpha\]
    is non-empty as well
\end{theorem}


\begin{figure}[!ht]
    \centering
    \resizebox{1\textwidth}{!}{%
    \begin{circuitikz}
    \tikzstyle{every node}=[font=\LARGE]
    
    
    
    
    \draw  (2,10.5) ellipse (1.25cm and 2.25cm);
    \draw  (4.75,10.5) ellipse (1.25cm and 2.25cm);
    \draw  (7.5,10.5) ellipse (1.25cm and 2.25cm);
    \draw  (11.5,10.5) ellipse (1.25cm and 2.25cm);
    
    \node [font=\normalsize] at (2, 13.2) {$A_0$};
    \node [font=\normalsize] at (4.75, 13.2) {$A_1$};
    \node [font=\normalsize] at (7.5, 13.2) {$A_2$};
    \node [font=\normalsize] at (11.5, 13.2) {$A_n$};
    
    \node [font=\LARGE] at (9.5,10.5) {...};
    \node [font=\LARGE] at (13.5,10.5) {...};
    
    
    \draw [short] (1.75,9.75) -- (4.75,9.75);
    \draw [short] (4.75,9.75) -- (7.5,11.5);
    \draw [short] (7.5,11.5) -- (11.5,10.25);
    
    \node [font=\small] at (1.75,9.75) {x};
    \node [font=\small] at (4.75,9.75) {x};
    \node [font=\small] at (7.5,11.5) {x};
    \node [font=\small] at (11.5,10.25) {x};
    \end{circuitikz}
    }%
    
    \label{fig:my_label}
    \end{figure}
    
\begin{remark}
    The Axiom of Choice is used to pick from a \textbf{non-finite} product of non-empty sets.
\end{remark}
Picking from a finite product is pretty straight forward, take $\mathbb{R}^n$ as an example.
\subsection{Orderings of Sets}
\begin{definition}[Partial Order]
    A \href{https://en.wikipedia.org/wiki/Partially_ordered_set}{Partial Order}, $\leq$, on a set $P$, is a binary relation that satisfies
    \begin{enumerate}
        \item(Reflexitivity) for all $p\in P$, $p\leq p$ 
        \item(Anti-Symmetry) for all $p,q\in P$ if $p\leq q$ and $q\leq p$, then $p=q$
        \item(Transitivity) for all $p,q,r\in P$ if $p\leq q$ and $q\leq r$, then $p\leq r$
    \end{enumerate}
    For any $p,q\in P$ if either $p\leq q$ or $q\leq p$ is always true, then we say that $\leq$ is total\footnote{or linear, but I will usually be using total}
\end{definition}
Let $(P,\leq)$ be a partially ordered set,
\begin{enumerate}
    \item $p\in P$ is maximal if there is no $q$ such that $p<q$
    \item $q\in P$ is minimal if there is no $p$ such that $p<q$
    \item $Q\subset P$ is bounded, if there exists an $r\in P$ such that $q\leq r$ for all $q\in Q$
    \item $C\subseteq P$ is a \href{https://proofwiki.org/wiki/Definition:Chain_(Order_Theory)}{Chain} if the restriction of $\leq$ to $C$ is total
\end{enumerate}
Further more we say that a totally ordered set $P$ is \href{https://en.wikipedia.org/wiki/Well-order}{Well-Ordered} if every non-empty subset has a minimal element.
\begin{theorem}[Well-Ordering Principle]
    Every non-empty set is well-orderable
\end{theorem}
\begin{proof}
    The proof of this requires the Axiom of Choice, the proof wasn't covered in class but you're interested \href{https://en.wikipedia.org/wiki/Well-ordering_theorem#Proof_from_axiom_of_choice}{heres a link}
\end{proof}
\begin{proposition}
    In a well ordered set $S$, every element, which is not maximal, has an immediate successor
\end{proposition}
\begin{remark}
    The intuition for this is that given a non maximal element, $x$, there exists nothing between $x$ and it's successor. or more formally, for $x\in S$, there exists $y\in S$ such that $x<y$, and if $x<z$ it follows that $y\leq z$
\end{remark}
\begin{proof}
    let $x\in S$ be an arbitrary non-maximal element, then $\{z\in S\vert x<z\}=A$ is a non-empty set. By Well-Ordering, let $y=\min{A}$, then we have that $y>x$ and for all $z\in A$, $y\leq z$, by definition.
\end{proof}
\subsection{Zorn's Lemma}
\begin{lemma}[Zorn's Lemma]
    Every non-empty partially ordered set in which every chain is bounded has a maximal element.
\end{lemma}
The proof for this is rather complicated and technical, much too far out of the scope for this course, but if you're interested you can look it up \href{https://pi.math.cornell.edu/~kbrown/6310/zorn.pdf}{here}. We're asked to prove a similiar result on the problem set, but we've been given the extra hypothesis of that every chain is finite. A rather surprising result is that in some axiomatic settings, Zorn's Lemma, the axiom of choice, and the well-ordering principle are all equivalent. You can again read about it \href{https://www.borisbukh.org/MathStudiesAlgebra1718/notes_ac.pdf}{here}.

\subsection{Order Topologies}
\begin{definition}[Order Topology]
    Given an ordered set $(X,<)$ with at least 2 elements, the order topology on $X$ is the topology generated by the basis consisting of intervals
    \begin{enumerate}
        \item $(x,y) = \{z\in X\vert x<z<y\}$
        \item $[x_{min},y) = \{z\in X\vert x_{min}\leq z< y\}$
        \item $(x,y_{max}] = \{z\in X\vert x< z\leq y_{max}\}$
    \end{enumerate}
    We notice that the minimum and maximum might not exist. If either exists we add 2 and 3 to our basis accordingly
\end{definition}
Before we delve into some examples, it is important to recall that a set $X$ is countable if there exists a surjection
\[f:\mathbb{N}\to X\]
We require that it is surjective so that finite sets are also countable.\footnote{It is convention that the empty set is also countable}
\begin{example}[The Line]
    The underlying set is $\mathbb{R}$, with the usual order. The order topology is generated by the open intervals of $\mathbb{R}$, as we have that $\mathbb{R}$ does not have a maximum
\end{example}
\begin{example}[The Natural Numbers]
    The underlying set are the Natural Numbers. the order is as usual. We notice that $\mathbb{N}$ isn't generated by the open intervals, because 0 doesn't exist within any open interval. But we have that the basis for the order topology of $\mathbb{N}$ is indeed a basis, because we first may let $n\in\mathbb{N}$ be arbitrary, if $n=0$ then $n\in[0,1)$, otherwise we have that $n\in(n-1,n+2)$, which are both sets that are in the basis. We further have that the order and discrete topologies on $\mathbb{N}$ are equal.
\end{example}
\begin{example}[Two Copies of the Natural Numbers]
    The underlying set is $\{0,1\}\times\mathbb{N}$, and the order is the lexicographic order. We actually have that every point in this topology is open, apart from $\langle1,0\rangle$
\end{example}

\begin{example}[The Lexicographic Plane]
    The underlying set is $\mathbb{R}^2$ and the order relation is given by
    \begin{center}
        $\langle a,b\rangle\footnote{We use this notation to represent a tuple, to avoid confusion with intervals}<_{lex}\langle c,d\rangle$ if either
        \begin{enumerate}
            \item $a<_\mathbb{R}c$
            \item $a=b$ and $b<_\mathbb{R}d$
        \end{enumerate}
    \end{center}
\end{example}

\begin{figure}[h]
    \centering
    \begin{tikzpicture}
        \draw[>=triangle 45, <->] (0,0) -- (0,-4.5);
        \draw[>=triangle 45, <->] (-0.5,-4) -- (4,-4);

        \node at (-6.2,0) {$y$};
        \node at (-1.9,-4.2) {$x$};
        
        \draw[red, thick, -Stealth] (-5.5,-4.2)   -- (-5.5,-0.4);
        \draw[red, thick, -Stealth] (-5,-4.2)   -- (-5,-0.4);
        \draw[red, thick, -Stealth] (-4.5,-4.2)   -- (-4.5,-0.4);
        \draw[red, thick, -Stealth] (-4,-4.2)   -- (-4,-0.4);
        \draw[red, thick, -Stealth] (-3.5,-4.2)   -- (-3.5,-0.4);
        \draw[red, thick, -Stealth] (-3,-4.2)   -- (-3,-0.4);


        \draw[>=triangle 45, <->] (-6,0) -- (-6,-4.5);
        \draw[>=triangle 45, <->] (-6.5,-4) -- (-2,-4);

        \node at (-0.2,0) {$y$};
        \node at (4.1,-4.2) {$x$};

        \draw[black, thick, dotted] (0.5,-4.2)   -- (0.5,-0.4);
        \draw[black, thick, dotted] (1.5,-4.2)   -- (1.5,-0.4);
        \draw[black, thick, dotted] (2,-4.2)   -- (2,-0.4);
        \draw[black, thick, dotted] (2.5,-4.2)   -- (2.5,-0.4);
        \draw[black, thick, dotted] (3,-4.2)   -- (3,-0.4);

        \node[rotate=90] at (0.5, -3.6) {(};
        \node[rotate=90] at (0.5, -1.2) {)};

        \node[rotate=90] at (1.5, -3.6) {(};
        \node[rotate=90] at (3, -2.2) {)};

        \path[pattern color = red, pattern=north west lines] (0.4,-3.6) -- (0.4,-1.2) -- (0.6,-1.2) -- (0.6,-3.6) -- cycle;

        \path[pattern color = red, pattern=north west lines] (1.4,-3.6) -- (1.4,-0.4) -- (1.6,-0.4) -- (1.6,-3.6) -- cycle;

        \path[pattern color = red, pattern=north west lines] (1.9,-4.2) -- (1.9,-0.4) -- (2.1,-0.4) -- (2.1,-4.2) -- cycle;

        \path[pattern color = red, pattern=north west lines] (2.4,-4.2) -- (2.4,-0.4) -- (2.6,-0.4) -- (2.6,-4.2) -- cycle;

        \path[pattern color = red, pattern=north west lines] (2.9,-4.2) -- (2.9,-2.2) -- (3.1,-2.2) -- (3.1,-4.2) -- cycle;
    \end{tikzpicture}
    \caption{The arrows in the left diagram are to represent the order, and the shaded regions on the right are the open sets.}
\end{figure}

\begin{example}[$\omega_1$, The Smallest Uncountable Ordinal]
    This is an uncountable, well ordered set such that $\forall\alpha\in\omega_1$, the set $\{\gamma\vert\gamma<\alpha\}$ is countable
\end{example}
\begin{theorem}
    $\omega_1$ exists
\end{theorem}
\begin{proof}
    Let $X$ be an uncountable set and let $\leq$ be a well ordering on $X$. Then consider the lexicographic order in $\{0,1\}\times X$. We have the following two facts
    \begin{itemize}
        \item $\{x\in\{0,1\}\times X\vert \{y\leq_{lex}x\}\text{ is uncountable}\}$ is non empty.
        
        This is true because $\langle 1,\min X\rangle$ exists within the set

        \item By Well-Ordering, there exists a minimum, $x$, with uncountably many predecessors, so the set of predecessors of $x$ is uncountable.
    \end{itemize}
\end{proof}